\section{Testing}

To be able to test and compare the SIMD processor to the normal SISD a program
needed to be written. This program needed to be able to be run with little or no
difference using both either instruction sets.

Two algorithms were thought of that could make use of the SIMD architecture.
The first one is ghosting two images together, the other is to run a fast
Fourier transform. Because the ghosting was very simple to implement and had a
visual output, it was chosen to be implemented first.

\subsection{Ghosting}
  Ghosting two images is where the value of each pixel location of each image is
  added with a 50\% opacity each. To keep the process simple the images to be
  used for testing are single channel eight bit grey scale images. These images
  are first loaded into memory via a Universal asynchronous receiver/transmitter
  (UART) then processed by the CPU, then once processed they are sent back to a
  computer to be viewed via the UART.
  
  % TODO: INsert a bunch of ghosting images..

  The ghosting algorithm process involves two ALU operations, the first one is
  to add the two pixel values, which means they are converted into a 9 bit
  number from the carry. This works fine as the 16bit architecture of Leros can
  store the 9 bit number is a 16 bit register. Then once the pixels have been
  added, the pixel value is divided by two. This has the same effect of giving
  both the images 50\% opacity. To do this divide, a simple right shift is
  applied to the pixel ignoring the least significant bit.

