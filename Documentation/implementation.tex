\section{Design \& Implementation}
\subsection{Instruction Set Extension}
The Leros microcontroller has a restricted instruction set, lacking multiply
or divide instructions. A multiplication instruction was required for the
project's selected application (TODO: was it??), therefore the Leros code
had to
be changed. This was complicated by the structure of Leros' instruction
register, which did not have room for the additional logic to accommodate
new
operations. Since the Leros code supports only two arithmetic instructions,
a single bit in the opcode was used to determine whether the instruction
was an ADD or a SUB instruction.

The first 5 bits of the instruction in the Leros CPU
corresponded to that shown
in Table \ref{tab:original-instruction}.
\begin{table}
\begin{tabular}{|p{0.2\textwidth}|p{0.2\textwidth}|p{0.2\textwidth}|}
\hline
\textbf{Bits} & Meaning \\
00000 & nop \\
00001 & arithmetic operation (add/sub) \\
00010 & right shift \\
00011 & reserved \\
00100 & alu / logic ops \\
00101 & load \\
00110 & store \\
00111 & IO \\
\end{tabular}
\end{table}

Whether the arithmetic instruction was an ADD or SUB was further determined by
the second bit of the instruction. It was decided that this was inefficient and
limiting. The instruction format was changed such that the first 5 bits of the
instruction simply set a flag, and the actual operation was encoded in bits 1
and 2 of the instruction. The flag was defined as a new type and takes one of
three values: \texttt{arith\_flag} for arithmetic operations,
\texttt{logic\_flag} for logic operations, and  
\texttt{io\_flag} for IO. This also left space for an additional four
operations should they ever be required.

Arranging things thus allowed many more instructions to be included. For each
class of instruction, four operations were possible, giving a total of
$4 \times 4 = 16$ operations. This also left space for an additional four
operations should they ever be required. Additionally, it  had the added
benefit of allowing left and
right shift and rotate instructions to be added. The Leros design only supports
right-shift; the reason for the omissions of the other operations is unknown.

A major structural change was made in order to accommodate the additional
instructions. 

\subsection{Vector ALU}
\subsection{Assembler for VHDL}

  Leros provided us with an assembler, but it was shit.
