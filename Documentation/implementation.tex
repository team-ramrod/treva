\section{Design \& Implementation}
\subsection{Instruction Set Extension}
The Leros microcontroller has a restricted instruction set, lacking multiply
or divide instructions, and shift and rotate operations. A multiplication
instruction was originally required for the
project's selected application (TODO: was it??), and thereforerequired a
change to Leros' code. Such a change was complicated by the structure of
Leros' instruction
register, which did not have room for the additional logic to accommodate
new
operations. Since the Leros code supports only two arithmetic instructions,
a single bit in the opcode was used to determine whether the instruction
was an ADD or a SUB instruction.

The first 5 bits of the instruction in the Leros CPU
corresponded to that shown
in Table \ref{tab:original-instruction}.

\begin{table}
\centering
\caption{The original Leros instruction categories}
\label{tab:original-instruction}
\begin{tabular}{|l|l|}
\hline
\textbf{Bits} & Meaning \\
\hline
\texttt{00000} & NOP \\
\texttt{00001} & Arithmetic Operation (ADD/SUB) \\
\texttt{00010} & Right Shift \\
\texttt{00011} & reserved \\
\texttt{00100} & ALU / Logic Operation \\
\texttt{00101} & Load \\
\texttt{00110} & Store \\
\texttt{00111} & IO \\
\hline
\end{tabular}
\end{table}

Whether the arithmetic instruction was an ADD or SUB was further determined by
the second bit of the instruction. It was decided that this was inefficient and
limiting. However, it was desirable than any changes to the opcode preserved the
majority of the existing form. The instruction format was changed such that the
first 5 bits of the instruction simply set a flag, and the actual operation was
encoded in bits 1 and 2 of the instruction. This is in contrast to the original
Leros code which encodes the arithmetic instruction in one bit, but was limited
to two such instructions. The flag set was defined as a new type and takes one
of three values: \texttt{arith\_flag} for arithmetic operations,
\texttt{logic\_flag} for logic operations, and  \texttt{io\_flag} for IO. This
also left space for an additional four operations should they ever be required.
The additional bit needed to represent the new instructions was obtained by using
one of the bits previously used to represent a logic opeartion, which themselves
were now represented using the flags mentioned abouve.

Arranging things thus allowed many more instructions to be included. For each
class of instruction, four operations were possible, giving a total of $4 \times
4 = 16$ operations. This also left space for an additional four operations
should they ever be required. Additionally, it  had the added benefit of
allowing left and right shift and rotate instructions to be added. The Leros
design only supports right-shift; the reason for the original omissions of the other
operations is unknown.



\subsection{Vector ALU}
\subsection{Assembler for VHDL}

  Leros provided us with an assembler, this was based on an ANother Tool for
  Language Recognition (ANTLR) generated Java parser and lexer.  Because of the
  quite major changes that would be required to add in support for vector
  operations, along with no one in the group having experience with ANTLR, it
  was decided that re-writing an assembler in a known language would be useful.

  The re-write was performed using Ruby, a two-pass assembler was created that
  matched the output of the Leros assembler, with one slight difference.  The
  Leros assembler output a VHDL file to synthesis as a ROM, the new assembler
  output a binary file.  To keep the integrated ROM a VHDLiser was also written
  that input the binary and output a VHDL ROM.
