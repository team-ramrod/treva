\section{Introduction}

\IEEEPARstart{T}{he} Single Instruction, Multiple Data (SIMD) architecture has been in 
existence since the 1970s with implementations in the supercomputers of Cray 
and its contemporaries. Modern versions play a crucial role in the 
desktop computer market, where the power SIMD brings to signal
processing affords users fast multimedia decoding\cite{bonnot2008definition}. This
report will detail exactly how SIMD is implemented and demonstrate its usefulness
in new CPU cores.  This will be done in the context of the
TREVA\footnote{\url{https://github.com/team-ramrod/treva}} project --
\emph{Team-Ramrod's ENEL429 VHDL Assignment}.



%TODO: Introduce what's new in SIMD then what's in our report

To test the SIMD core  and compare it to an SISD core, test applications are
needed. One of the possible applications that was being explored was the
implementation of a Fast Fourier Transform \cite{Jamieson198648}. The other
application is ghosting two images, which, while not as useful for real applications, would
produce valuable metrics. Ghosting can be thought of as overlaying one image
on top of another with 50\% opacity, or averaging the images. This also produces a visual output
that can be inspected for errors.


\subsection{Spartan 3E-1600E}
% Talk about the board/FPGA we used
The VHDL code was synthesized on the Spartan 3E-1600 Development Board
provided. This development board hosts a Xilinx XC3S1600E FPGA with 1.6 million
logic gates. The board also contains external peripherals
including UART, 10-100 Ethernet, Flash memory, DDR memory, LCD display, PS/2
mouse or keyboard port, buttons, LEDs and switches. The Spartan family is targeted
at low cost, high volume applications and as such Xilinx has produced several hundred million
of the devices\cite{xilinxpress} since their launch in 1998.



