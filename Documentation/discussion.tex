\section{Discussion}
%TODO: everything
In its current state the designed SIMD processor has several drawbacks that mean
it does not perform at the level that was strived for. The limit of 32 ALUs
imposed by memory issues and the additional memory operations required due to
the ALUs not sharing memory has caused these serious performance constraints.
While the SIMD processor would still easily out perform an SISD implementation
based on the same design, it did not achieve the high levels of throughput that
would justify the use of the FPGA implementation for use in
image processing and potentially FFT applications.  

The source of many of the issues that were encountered stemmed from the original
Leros architecture that the processor was designed on. While when the processor
was first chosen it appeared to have a solid implementation, the more the
architecture was examined the more questionable design decisions were
encountered. The reason for these problems most likely stems from the
large push on the CPU to minimize its size at all costs. This, coupled with the
design and all the code being written by a single person, meant issues were
encountered when an attempt to understand and modify it was made.

The two
largest issues were that the UART code that came with the CPU was not
already implemented in the CPU, and when added in did not appear to function
correctly. The second issue was that while the block diagram of the system showed
a large number of small components each linked to the others in a logical
arrangement, large amounts of the units are implemented inside single
processes in the code. An example of this is in the code the ALU, input
multiplexer, memory and accumulator are all defined at once in the same process
with a large amount of coupling through shared signals.

\subsection{Effects of Single-Cycle Instructions}
The Leros code base was designed for single clock cycle instructions. Such a
restriction poses few problems for addition and subtraction instructions, but
there was the possibility that any further added operations may not be able to be
completed in a single cycle. This presents a problem because Leros' pipelined
design relies on these single cycle instructions.

It was originally thought that the goals of the project would benefit from
multiply and divide instructions. There was no way to control whether these
would be implemented as single-cycle, as this would be decided during the
synthesis stage by the Xilinx compiler. Fortunately, after implementation of the
multiplication operation, inspection of the RTL schematic showed that this
instruction had indeed been implemented as single-cycle.

Division was not expected to be implemented as a single-cycle instruction by
Xilinx, and this suspicion was confirmed on inspection of the RTL schematic
after a division instruction was added. For this reason, no division instruction
was included in the final design, in order to ensure that the pipeline
architecture was not compromised. The lack of a division instruction did not
hinder the project, however it is worth noting this deficiency.
